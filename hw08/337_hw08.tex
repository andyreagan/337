\documentclass[11pt]{article}
\usepackage{amsmath,amsthm,amssymb,fullpage,graphicx,hyperref,listings}
\usepackage{listings,color,setspace}
\author{Andy Reagan}
\title{Math 337 Homework 08}

     \def\NN{\mathbb{N} }
     \def\ZZ{\mathbb{Z} }
     \def\QQ{\mathbb{Q} }
     \def\RR{\mathbb{R} }
     \def\CC{\mathbb{C} }
     \def\f{\frac }
     \def\b{\begin }
     \def\e{\end }
     \def\Log{\text{Log} \,}
     \def\Re{\text{Re} \, }

\lstset{language=MATLAB,
basicstyle=\ttfamily\scriptsize\singlespacing,
keywordstyle=\color{blue},
stringstyle=\color{red},
commentstyle=\color{green},
morecomment=[l][\color{magenta}]{\#},
frame=L,
xleftmargin=\parindent,
%%numbers=left,                   %% where to put the line-numbers
%%numberstyle=\scriptsize,      %% the size of the fonts that are used for the line-numbers
%%stepnumber=1,                   %% the step between two line-numbers. If it is 1 each line will be numbered
numbersep=5pt,
breaklines=true,        %% sets automatic line breaking
breakatwhitespace=false,    %% sets if automatic breaks should only happen at whitespace
escapeinside={\%*}{*)} 
}


     \newcommand{\pdiff}[2]{\frac{\partial #1}{\partial #2}}
     \newcommand{\partialdiff}[2]{\frac{\partial #1}{\partial #2}}
     \newcommand{\pdiffsq}[2]{\frac{\partial^2 #1}{{\partial #2}^2}}
     \newcommand{\pdiffcu}[2]{\frac{\partial^3 #1}{{\partial #2}^3}}
     \newcommand{\pdiffhi}[3]{\frac{\partial^#3 #1}{{\partial #2}^#3}}
     \newcommand{\diff}[2]{\frac{{\rm d}#1}{{\rm d}#2}}
     \newcommand{\diffsq}[2]{\frac{{\rm d}^{2}#1}{{\rm d} {#2}^2}}
     \newcommand{\diffhi}[3]{\frac{{\rm d}^#3 #1}{{\rm d} {#2}^#3}}
     \newcommand{\tdiff}[2]{\mbox{d} #1/\mbox{d} #2}
     \newcommand{\tdiffsq}[2]{\mbox{d}^{2} #1/\mbox{d} {#2}^2}
     \newcommand{\tpdiff}[2]{\partial #1/\partial #2}
     \newcommand{\tpdiffsq}[2]{\partial^2 #1/\partial {#2}^2}
     \newcommand{\bvec}[1]{\vec{ {\bf #1 } }}
     \newcommand{\oh}[1]{O(h^{{#1}})}

\begin{document}
\maketitle

\begin{enumerate}

\item Use the Gerschgorin Circles Theorem and the fact that the eigenvalues of real symmetric matrices are real to obtain the best estimate for the location of the eigenvalues of the following tri-diagonal matrix:
\[ A = \left ( \begin{array}{ccccccc} a & -1 & 0 & . & . & . & 0\\
-1 & a & -1 & 0 & . & . & 0\\
0 & -1 & a & -1 & 0 & . & 0\\
 &  &  & . & . & . & \\
0 & . & . & 0 & -1 & a & -1\\
0 & . & . & . & 0 & -1 & a\end{array} \right ) , \]
where $a$ is a real number.
In particular, what is the minimum distance between an eigenvalue of this matrix and zero?

\bigskip
\textbf{Solution:} By the Gerschgorin Circles Theorem, all of the eigenvalues are found in circles centered at $a$.
The radius of these circles is $1$, corresponding to the first and last rows, and $2$ otherwise.
Since the eigenvalues are real, we know that they fall in the interval $[a-2,a+2]$ on the real line.

In particular, the distance between the eigenvalues of this matrix and 0 is at least $a-2$ if $a>2$ and $a+2$ is $a<2$. If $|a|\leq 2$, the minimum distance is 0 (i.e., the eigenvalues can well be 0).

%% My code and a solution plot follow.

%% \lstinputlisting[language=Matlab]{andy_hw08_prb01.m}

%% \begin{figure}[h!]
%%   \centering
%%     \includegraphics[width=0.5\textwidth]{andy_hw08_prb01_01.pdf}
%%   \caption{Solution of the BVP with the shooting method.}
%% \end{figure}

\item Consider a linear BVP
\[ y'' + 2(2-x)y' = 2(2-x), y(0) = -1, y(6) = 5.\]
Discretize it using scheme (8.4) with $h = 1$.
\begin{enumerate}

\item[(i)] Verify that you obtain a linear system
\[ \left ( \begin{array}{ccccc} -2 & 2 & 0 & 0 & 0\\
1 & -2 & 1 & 0 & 0\\
0 & 2 & -2 & 0 & 0\\
0 & 0 & 3 & -2 & -1\\
0 & 0 & 0 & 4 & -2\end{array} \right ) \left (\begin{array}{c} Y_1 \\
Y_2\\
Y_3\\
Y_4\\
Y_5\end{array} \right ) = \left (\begin{array}{r} 2 \\
0\\
-2\\
-4\\
4\end{array} \right ).\]

\item[(ii)] Solve is using MATLAB. What do you obtain?
\item[(iii)] The result you have obtain in part (ii) occurs because one of the conditions of Theorem 8.3 is violated.
What is the condition?
\end{enumerate}

\bigskip
\textbf{Solution:} 
\begin{enumerate}
\item[(i)] For this problem, equation (8.4) becomes
\begin{align} &Y_0 = -1;\\
&(1+h(2-x_n))Y_{n+1} -2 Y_n + (1-h(2-x_n)) Y_{n-1} = 2 h^2 (2-x_n),~~~1\leq n \leq 5 \label{eq:11}\\
&Y_6 = 5.\end{align}
Setting $h = 1$ in \eqref{eq:11}, this become
\begin{equation} (3-x_n)Y_{n+1} -2 Y_n + (-1+x_n) Y_{n-1} = 4-2x_n; \label{eq:11}\end{equation}
We have the following 5 discrete equations for $Y_{1\ldots 5}$, where $Y_0$ and $Y_6$ are the supplied BC.
\begin{align*} (3-x_1)Y_{2} -2 Y_1 + (-1+x_1) Y_{0} = 4-2x_1; \\
(3-x_2)Y_{3} -2 Y_2 + (-1+x_2) Y_{1} = 4-2x_2; \\
(3-x_3)Y_{4} -2 Y_3 + (-1+x_3) Y_{2} = 4-2x_3; \\
(3-x_4)Y_{5} -2 Y_4 + (-1+x_4) Y_{3} = 4-2x_4; \\
(3-x_5)Y_{6} -2 Y_5 + (-1+x_5) Y_{4} = 4-2x_5. \end{align*}
Plugging in $x_i = i$ and $Y_0 = -1, Y_6 = 5$, and moving constants to the RHS, we are left
\begin{align*} 2Y_{2} -2 Y_1 = 2; \\
Y_{3} -2 Y_2 + Y_{1} = 0; \\
-2 Y_3 + 2Y_{2} = -2; \\
-Y_{5} -2 Y_4 + 3Y_{3} = -4; \\
-2 Y_5 + 4Y_{4} = 4. \end{align*}
Intuitively obvious to the casual observer, this agrees with the desired linear system.

\item[(ii)] MATLAB obtain a vector of all NaN's, the matrix being singular to working precision.

\lstinputlisting[language=Matlab]{andy_hw08_prb02.m}

\item[(iii)] The requirement of Theorem 8.3 that $h\mathcal{P} \leq 2$ is violated.
Here $P(x)$ achieves a magnitude of 8, at $x = 6$, and we had chose $h = 1$.
Therefore, our $h\mathcal{P} = 8$.
\end{enumerate}

\item 
\begin{enumerate}
\item Give an operation count for finding $L$ and $U$ for a tridiagonal matrix, as per Eq. (8.21).
\item Give operation counts for solving the systems in (8.17), as per (8.22) and (8.23). 
\item Given the total operations count for the Thomas algorithm.
\end{enumerate}

\bigskip
\textbf{Solution:} 
\begin{enumerate}
\item We compute $M-1$ new $\alpha$ values, and $M-1$ new $\beta$ values.
Each new $\alpha$ requires one operation, and each $\beta$ requires 2.
Therefore, we use $3(M-1)$ operations.
\item Solving for $\vec{z}$ requires exactly $2(M-1)$ operations.
The solving for $\vec{y}$ requires exactly $1+3(M-1)$ operations.
In total, this is $1+5(M-1)$ operations.
\item The total operation count is (adding the two previous counts) $1+8(M-1)$.
\end{enumerate}

\item Use the \verb|thomas.m| function, posted under ``Codes for examples and selected homework problems,'' to solve a tridiagonal system $A\vec{y} = \vec{r}$ where $A$ has '2' on the main diagonal and '-1' on the two subdiagonals.
Take $\vec{r} = [1,-1,1,-1,\ldots]^T$ and $M = 1000$ and 5000.
Now solve the same system using Matlab's solver.
Here, you need to investigate {\em two} cases: One, when $A$ is constructed as a regular (i.e., full) matrix and two, when it is constructed as a sparse matrix.

Compare the computational times required to solve this system for your code and for the MATLAB's solver, in those two cases.
In particular, comment on {\em how the computational times scale with $M$} in each of the three cases considered.

\bigskip
\textbf{Solution:} I compare with the following code, and plot how the computational times scale with $M$ (using more values of $M$).
I find that both the \verb|thomas.m| code and MATLAB's sparse solver scale sub-linearly with $M$, with MATLAB's solver being faster.

For a non-sparse matrix, MATLAB's solver scales superlinearly with $M$, and becomes useless quickly.

\lstinputlisting[language=Matlab]{andy_hw08_prb04.m}

\begin{figure}[h!]
  \centering
    \includegraphics[width=0.5\textwidth]{andy_hw08_prb04_01.pdf}
  \caption{Scaling of computational time for thomas.m and MATLAB's built in solver, with a sparse matrix, versus $M$.}
\end{figure}

\begin{figure}[h!]
  \centering
    \includegraphics[width=0.5\textwidth]{andy_hw08_prb04_02.pdf}
  \caption{Scaling of computational time for MATLAB's built in solver, with a non-sparse matrix, versus $M$. It is impractical on a personal computer to extend $M$ much further than a 10,000 by 10,000 matrix, which takes 7 second to solve.}
\end{figure}

\item Redo Problem 4 of HW07 using the discretized BVP (8.4) with $h = 0.09$ (ste size $h = 0.1$ will not ``fit'' into the interval $[0,1.62]$).
Compare the result with that found in HW07.
Which method, shooting or finite-difference discretization, is preferable for solving BVPs like this one?
\begin{enumerate}
\item[Bonus (a)] Plot the error of your numerical solution. Explain the result.
\item[Bonus (b)] Repeat the problem with $h = 0.01$ and plot the error. Explain why it is greater than that for $h = 0.09$.
\end{enumerate}

\bigskip
\textbf{Solution:} The problem is
\begin{align*} y'' = 30^2 (y - 1 + 2x), ~~~~y(0) = 1, ~~~~y(1.62) = -2.24.\end{align*}
Discretizing using (8.4) is 
\begin{align*} &Y_0 = 1;\\
&Y_{n+1} - Y_n (2-30^2h^2) + Y_{n-1} = 30^2 h^2 (2x_n -1 ) , ~~~1\leq n \leq N-1;\\
&Y_N = -2.24\end{align*}
I build this matrix in MATLAB and solve it using the built in solver.

\lstinputlisting[language=Matlab]{andy_hw08_prb05.m}

\begin{figure}[h!]
  \centering
    \includegraphics[width=0.5\textwidth]{andy_hw08_prb05_01.pdf}
  \caption{Solution the HW07 problem 4, using the finite difference scheme given by Eq (8.4).}
\end{figure}

\item Solve the BVP
\[ (1+x)^2 y'' = 2y-4 , ~~~~y(0) =  0, ~~~~y(1) + 2y'(1) = 2 \]
using the second-order accurate discretization (8.4) (for $n = 1,\ldots,N-1$) of this BVP.
Use Method 1 of Sec. 8.4 modified in such a way that it can handle the mixed type BC at the {\em right} end point of the interval.

Confirm that you numerical solution has the second order of accurarcy by comparing it at different $h$ with the exact solution $y_\text{exact} = 2x/(1+x)$.
For this, do the following:
\begin{enumerate}
\item[(i)] Run your code with $h = 0.05$ and $h = 0.025$;
\item[(ii)] Plot the error as a function of $x$;
\item[(iii)] Confirm that the maximum error scales as $O(h^2)$.
\end{enumerate}

\bigskip
\textbf{Solution:} To account for the mixed BC,
\[ A_1 y(b) + A_2 y'(b) = \beta,\]
we introduce the point $Y_{N+1}$ and approximate $y'(b)$ with a second order approximation
\[y'(b) = \f{Y_{N+1}-Y_{N-1} }{2h} + \oh{2} .\]
The mixed BC is therefore discretized as
\begin{equation} A_1 Y_{N} + A_{2} \f{Y_{N+1}-Y_{N-1} }{2h} = \beta. \label{eq:disBC}\end{equation}
The ODE discretized at $x_N$ is
\begin{equation} \left ( 1+P_N \right ) Y_{N+1} - \left ( 2 - h^2 Q_N \right ) Y_N + \left ( 1 - \f{h}{2} P_N \right ) Y_{N-1} = h^2 R_{N} \label{eq:disODE} \end{equation}
Solving for $Y_{N+1}$ in \eqref{disBC} into \eqref{disODE} we have
\begin{equation} - \left ( 2 - h^2 Q_N + 2h \f{A_1}{A_2} \left ( 1 + \f{h}{2} P_N \right ) \right ) Y_N + 2 Y_{N-1} = h^2 R_{N}- \left ( 1+ \f{h}{2}P_N\right ) \f{2h\beta}{A_2}  \label{eq:disODEnew} \end{equation}
where the first two terms on the LHS and the last term on the RHS make these matrix diagonal entries different.
I solve this in MATLAB, and plot the error versus $h$.
We find that the maximum error increases fourfold when $h$ is doubled, confirming that the method is $\oh{2}$ accurate.

\lstinputlisting[language=Matlab]{andy_hw08_prb06.m}

\begin{figure}[h!]
  \centering
    \includegraphics[width=0.5\textwidth]{andy_hw08_prb06_01.pdf}
  \caption{Error in the solution to the BVP with mixed BC at $x = b$, using Method 1 of Sec 8.4, versus $h$.
           We confirm that the method is second order.}
\end{figure}

\item Show that if condition (8.42) and the two conditions stated one line below it hold, then the coefficient matrix in Method 2 based on Eq. (8.39) is SDD.

Bonus part: Equations (8.36) and (8.39) each lead to a second-order accurate method.
Therefore, solutions obtained by those methods must differ by $\oh{3}$.
Show {\em analytically} that this is indeed the case.

\bigskip
\textbf{Solution:} Condition (8.42) is that 
\[ A_1 A_2 \leq 0 \]
meaning that they are of opposite parity.
The other conditions ($Q \leq 0, h\mathcal{P} \leq 2$) will also be used.
Consider that the only change in the coefficient matrix from using Method 2, from the form of the original triadiagonal matrix (8.6) is the first row.

To retain SDD of the coefficient matrix, we therefore require that the entry in row 1, column 1 is greater than the entry in row 1, column 2.
Those entries are the coefficients on $Y_{0,1}$, which appear in equation (8.39).
Taking the hint, I begin by multiplying (8.39) by $h/A_2$.
This yeilds:


\end{enumerate}

\clearpage
\pagebreak
{\huge Appendix 1: ODE Functions}
%% \lstinputlisting[language=Matlab]{andy_hw08_prb01_ODE.m}
%% \lstinputlisting[language=Matlab]{andy_hw08_prb01_ODEh.m}

\clearpage
\pagebreak
{\huge Appendix 2: Numerical Methods}

%% \lstinputlisting[language=Matlab]{andy_ME.m}
%% \lstinputlisting[language=Matlab]{andy_MEr.m}

\end{document}
