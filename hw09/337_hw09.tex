\documentclass[11pt]{article}
\usepackage{amsmath,amsthm,amssymb,fullpage,graphicx,hyperref,listings}
\usepackage{listings,color,setspace}
\author{Andy Reagan}
\title{Math 337 Homework 09}

     \def\NN{\mathbb{N} }
     \def\ZZ{\mathbb{Z} }
     \def\QQ{\mathbb{Q} }
     \def\RR{\mathbb{R} }
     \def\CC{\mathbb{C} }
     \def\f{\frac }
     \def\b{\begin }
     \def\e{\end }
     \def\Log{\text{Log} \,}
     \def\Re{\text{Re} \, }

\lstset{language=MATLAB,
basicstyle=\ttfamily\scriptsize\singlespacing,
keywordstyle=\color{black},
stringstyle=\color{black},
commentstyle=\color{black},
morecomment=[l][\color{black}]{\#},
frame=L,
xleftmargin=\parindent,
%%numbers=left,                   %% where to put the line-numbers
%%numberstyle=\scriptsize,      %% the size of the fonts that are used for the line-numbers
%%stepnumber=1,                   %% the step between two line-numbers. If it is 1 each line will be numbered
numbersep=5pt,
breaklines=true,        %% sets automatic line breaking
breakatwhitespace=false,    %% sets if automatic breaks should only happen at whitespace
escapeinside={\%*}{*)} 
}


     \newcommand{\pdiff}[2]{\frac{\partial #1}{\partial #2}}
     \newcommand{\partialdiff}[2]{\frac{\partial #1}{\partial #2}}
     \newcommand{\pdiffsq}[2]{\frac{\partial^2 #1}{{\partial #2}^2}}
     \newcommand{\pdiffcu}[2]{\frac{\partial^3 #1}{{\partial #2}^3}}
     \newcommand{\pdiffhi}[3]{\frac{\partial^#3 #1}{{\partial #2}^#3}}
     \newcommand{\diff}[2]{\frac{{\rm d}#1}{{\rm d}#2}}
     \newcommand{\diffsq}[2]{\frac{{\rm d}^{2}#1}{{\rm d} {#2}^2}}
     \newcommand{\diffhi}[3]{\frac{{\rm d}^#3 #1}{{\rm d} {#2}^#3}}
     \newcommand{\tdiff}[2]{\mbox{d} #1/\mbox{d} #2}
     \newcommand{\tdiffsq}[2]{\mbox{d}^{2} #1/\mbox{d} {#2}^2}
     \newcommand{\tpdiff}[2]{\partial #1/\partial #2}
     \newcommand{\tpdiffsq}[2]{\partial^2 #1/\partial {#2}^2}
     \newcommand{\bvec}[1]{\vec{ {\bf #1 } }}
     \newcommand{\oh}[1]{O(h^{{#1}})}

\begin{document}
\maketitle

\begin{enumerate}

\item Solve the BVP
\[ y'' - \f{2y}{(1+x)^2} = -\f{4}{(1+x)^2}, ~~~~~y(0)=0,~~~y(1) = 1\]
using the collocation method with $\phi _j = \sin (j\pi x)$ for $j = 1,\ldots,M$ and $M=10$.
Use the equidistant collocation points $x_k = x_0 + k\cdot h$.
Compare you finite-element solution with the exact solution $y_{\text{exact}} = 2x/(1+x)$ by plotting them together, and also by plotting, in a separate figure, the error for $x\in [0,1]$.

\bigskip
\textbf{Solution:} 

\item Obtain Eqs. (9.20) and (9.22) of the notes.

\bigskip
\textbf{Solution:} 

\item Solve the BVP in Problem 1 by the Galerkin method with the hat functions and $M=10$.
Compare your result with the exact solution.
As in Problem 1, investigate how the error of the Galerkin method scales with $(1/M)$.

\bigskip
\textbf{Solution:} 

\end{enumerate}

\clearpage
\pagebreak
{\huge Appendix 1: ODE Functions}
%% \lstinputlisting[language=Matlab]{andy_hw07_prb05_ODE.m}
%% \lstinputlisting[language=Matlab]{andy_hw08_prb01_ODEh.m}

\clearpage
\pagebreak
{\huge Appendix 2: Numerical Methods}

%% \lstinputlisting[language=Matlab]{andy_hw08_prb08_r.m}
%% \lstinputlisting[language=Matlab]{andy_MEr.m}

\end{document}



